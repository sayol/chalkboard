\documentclass[aspectratio=169,12pt,t]{beamer}
\usetheme{ChalkBoard}
\newfontfamily{\bpfamily}{Baphnom}
\begin{document}
\begin{frame}{លីមីតត្រៀមបាក់ឌុប}
\setcounter{theorem}{4}
\begin{example}
	\itshape\bpfamily គណនាលីមីតខាងក្រោម៖
	\begin{enumerate}[a]
		\item $ \lim\limits_{x\to 2}\dfrac{x^3-2x^2-x+2}{x^2-5x+6} $
		\item $ \lim\limits_{x\to 0}\dfrac{\tan 3x}{5x} $
		\item $ \lim\limits_{x\to 0}\dfrac{\sin 2x-2\sin x}{x^2\sin x} $
	\end{enumerate}
\end{example}
\end{frame}
%
\begin{frame}[allowframebreaks]{ដំណោះស្រាយ}
\begin{enumerate}[a]
	\item $ \lim\limits_{x\to 2}\dfrac{x^3-2x^2-x+2}{x^2-5x+6} $ រាងមិនកំណត់ $ \frac{0}{0} $
	\begin{align*}
	\lim\limits_{x\to 2}\dfrac{x^3-2x^2-x+2}{x^2-5x+6}
	&=\lim\limits_{x\to 2}\dfrac{x^2(x-2)-(x-2)}{x^2-2x-3x+6}\\
	&=\lim\limits_{x\to 2}\dfrac{(x-2)(x^2-1)}{x(x-2)-3(x-2)}\\
	&=\lim\limits_{x\to 2}\dfrac{\cancel{(x-2)}(x^2-1)}{\cancel{(x-2)}(x-3)}\\
	&=\lim\limits_{x\to 2}\dfrac{x^2-1}{x-3}\\
	&=\frac{2^2-1}{2-3}=\frac{3}{-1}=-3
	\end{align*}
	\item $ \lim\limits_{x\to 0}\dfrac{\tan 3x}{5x} $ រាងមិនកំណត់ $ \frac{0}{0} $
	\begin{align*}
	\lim\limits_{x\to 0}\dfrac{\tan 3x}{5x}
	&=\lim\limits_{x\to 0}\left (\dfrac{\tan 3x}{3x}\times\frac{3}{5}\right )
	=1\times \frac{3}{5}=\frac{3}{5}\\
	&\textnormal{ ព្រោះ }\lim\limits_{u\to 0}\frac{\tan u}{u}=1\textnormal{ ដែល }u=3x
	\end{align*}
	\item $ \lim\limits_{x\to 0}\dfrac{\sin 2x-2\sin x}{x^2\sin x} $ រាងមិនកំណត់ $ \frac{0}{0} $
	\vskip-0.5\baselineskip
	\begin{align*}
	\lim\limits_{x\to 0}\dfrac{\sin 2x-2\sin x}{x^2\sin x}
	&=\lim\limits_{x\to 0}\dfrac{2\sin x\cos x-2\sin x}{x^2\sin x}\\
	&=\lim\limits_{x\to 0}\frac{2\cancel{\sin x}(\cos x-1)}{x^2\cancel{\sin x}}\\
	&=\lim\limits_{x\to 0}\frac{2(\cos x-1)}{x^2}\times\frac{(\cos x+1)}{\cos x+1}\\
	&=\lim\limits_{x\to 0}\frac{2(\cos^2 x-1)}{x^2(\cos x+1)}\\
	&=\lim\limits_{x\to 0}\frac{2(-\sin^2 x)}{x^2(\cos x+1)}\\
	&=\lim\limits_{x\to 0}\left[ \frac{\sin^2 x}{x^2}\times\frac{-2}{\cos x+1} \right]\\
	&=\lim\limits_{x\to 0}\left[ \left (\frac{\sin x}{x}\right )^2\times\frac{-2}{\cos x+1} \right]\\
	&=(1)^2\times\left(\frac{-2}{1+1}\right)
	=-1\;\textnormal{ ព្រោះ }\lim\limits_{x\to 0}\frac{\sin x}{x}=1
	\end{align*}
\end{enumerate}
\end{frame}
\end{document}