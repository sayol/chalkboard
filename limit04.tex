\documentclass[aspectratio=169,12pt,t,green]{beamer}
\usetheme{ChalkBoard}

\begin{document}
\begin{frame}{លីមីតត្រៀមបាក់ឌុប}
\setcounter{theorem}{3}
\begin{example}
	គណនាលីមីតខាងក្រោម៖
	\begin{enumerate}[a]
		\item $ \lim\limits_{x\to 1}\dfrac{-2x^3-2x^2+5x-1}{x^2+2x-3} $
		\item $ \lim\limits_{x\to 0}\dfrac{\sin 2x}{-3x} $
		\item $ \lim\limits_{x\to 1}\dfrac{\sin(\pi x)}{x^2-1} $
	\end{enumerate}
\end{example}
\end{frame}
%
\begin{frame}[allowframebreaks]{ដំណោះស្រាយ}
\begin{enumerate}[a]
	\item $ \lim\limits_{x\to 1}\dfrac{-2x^3-2x^2+5x-1}{x^2+2x-3} $ រាងមិនកំណត់ $ \frac{0}{0} $
	\begin{align*}
		\lim\limits_{x\to 1}\dfrac{-2x^3-2x^2+5x-1}{x^2+2x-3}
		&=\lim\limits_{x\to 1}\dfrac{-2x^3+2x^2-4x^2+4x+x-1}{x^2-x+3x-3}\\
		&=\lim\limits_{x\to 1}\frac{-2x^2(x-1)-4x(x-1)+(x-1)}{x(x-1)+3(x-1)}\\
		&=\lim\limits_{x\to 1}\frac{\cancel{(x-1)}(-2x^2-4x+1)}{\cancel{(x-1)}(x+3)}\\
		&=\lim\limits_{x\to 1}\frac{-2x^2-4x+1}{x+3}
		=\frac{-2-4+1}{1+3}=-\frac{5}{4}
	\end{align*}
	\item $ \lim\limits_{x\to 0}\dfrac{\sin 2x}{-3x} $ រាងមិនកំណត់ $ \frac{0}{0} $
	\begin{align*}
	\lim\limits_{x\to 0}\dfrac{\sin 2x}{-3x}
	&=\lim\limits_{x\to 0}\left (\dfrac{\sin 2x}{2x}\times\frac{2}{-3}\right )
	=(1)\left(-\frac{2}{3}\right)=-\frac{2}{3}\\
	&\text{ ព្រោះ }\lim\limits_{u\to 0}\frac{\sin u}{u}=1\text{ ដែល }u=2x
	\end{align*}
	\item $ \lim\limits_{x\to 1}\dfrac{\sin(\pi x)}{x^2-1} $ រាងមិនកំណត់ $ \frac{0}{0} $
	\vskip-\baselineskip
	\begin{align*}
	\lim\limits_{x\to 1}\dfrac{\sin(\pi x)}{x^2-1}
	&=\lim\limits_{x\to 1}\dfrac{\sin(\pi-\pi x)}{(x-1)(x+1)} && \text{ ព្រោះ }\sin(\pi-\alpha)=\sin\alpha\\
	&=\lim\limits_{x\to 1}\left [\dfrac{\sin(\pi-\pi x)}{\pi(x-1)}\times\frac{\pi}{x+1}\right ]\\
	&=\lim\limits_{x\to 1}\left [\dfrac{\sin(\pi-\pi x)}{\pi-\pi x}\times\frac{-\pi}{x+1}\right ]\\
	&=(1)\left(-\frac{\pi}{1+1}\right)
	=-\frac{\pi}{2} && \text{ ព្រោះ }\lim\limits_{u\to 0}\frac{\sin u}{u}=1
	\end{align*}
\end{enumerate}
\end{frame}
\end{document}